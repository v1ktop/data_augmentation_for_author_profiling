\chapter{Trabajo Relacionado}


Descubrir los rasgos de un autor anónimo es de interés para la comunidad científica en procesamiento de lenguaje natural. Existen numerosas razones, una de ellas es aprovechar el constante flujo de información en redes sociales para entender mejor el lenguaje coloquial de uso diario, hacer que nuestras máquinas puedan identificar emociones, estados de ánimo, el género y edad del usuario, etc. Muchos esfuerzos y avances se han realizado en la última década. 

En este capítulo se presenta, por un lado, el trabajo previo en área de perfilado de autor, en específico, los foros de evaluación PAN@CLEF y eRISK. Por otro lado, se presenta una revisión general del aumento de datos en tareas de clasificación de textos. Cabe aclarar que no existen estudios específicamente orientados al aumento de datos en la tarea de perfilado de autor. 


\section{Perfilado de autor}
Los trabajos en perfilado de autor se han enfocado a identificar diversos rasgos de los autores: género, edad, nivel educativo, ocupación, rasgos de personalidad, tendencia política. Incluso han ido más allá al tratar de determinar características de comportamiento y condiciones médicas (trastornos como la anorexia o la depresión clínica). 

%Diversas investigaciones se han realizado sobre la identificación de rasgos de un autor a partir de  sus escritos. 
Los primeros trabajos, motivados por la sociolingüística, utilizaron documentos formales: libros, ensayos y/o noticias; variando el tamaño del corpus estudiado de docenas a cientos de documentos. 
Uno de los primeros trabajos en perfilado de autor, usando medios automáticos, fue presentado por \citep{Pennebaker2002}. Los investigadores presentan evidencia que liga el uso de las palabras con aspectos de personalidad, situaciones sociales y psicológicas.

\citep{Argamon2009} demostró que se puede conocer el género, edad, lengua nativa y personalidad con un buen margen de exactitud, a través de ensayos personales de estudiantes. Las características relevantes encontradas fueron estilísticas, por ejemplo, el uso de pronombres, preposiciones y verbos modales. 
%En el caso de clasificar un texto como neurótico, la mayoría del contenido fue irrelevante y más útil características relacionadas con problemas personales y actividades de relajación.

En la actualidad, la investigación se ha enfocado determinar el perfil del autor utilizando datos extraídos de redes sociales, blogs y foros en línea. A continuación se presentan los trabajos presentados en dos foros de gran importancia en el área. 

\subsubsection{PAN@CLEF}
El mayor evento anual en perfilado de autor PAN es parte de las competencias organizadas bajo el marco del CLEF \citep{Rangel2013b, Rangel2019}. En este evento se ha estudiado el perfilado de autor desde una perspectiva multi-idioma, siendo el idioma inglés y español los más frecuentes. Las características recurrentes a perfilar ha sido género, edad \citep{Rangel2013b, Rangel2019, Rangel2016b} y personalidad \citep{Stammatatos2015}.
La mayoría de los trabajos se distingue por (i) el pre-procesamiento, (ii) la extracción de características o (iii) el método de clasificación.

La técnica más común de \textbf{pre-procesamiento} entre los participantes es remover o enmascarar elementos específicos de las redes sociales (\textit{hashtags}, menciones de usuario, enlaces a páginas web, emoticones) \citep{daneshvar2018gender, jimenez2019bots, Pizarro2019}. Además de convertir las palabras a minúsculas, utilizar lematización  o \textit{stemming}; algunos participantes remueven puntuación, palabras de paro y caracteres especiales.

En cuanto a la \textbf{extracción de características} los n-gramas de caracteres y palabras son ampliamente usados, en efecto las mejores soluciones propuestas para el perfilado de género en el PAN 2017, 2018 y 2019 \citep{basile2017there, daneshvar2018gender, Pizarro2019} utilizaron un ensamble de n-gramas de caracteres y n-gramas de palabras. En estos trabajos se ha identificado que una representación mediante n-gramas de caracteres puede ser capaz de capturar fragmentos relacionados a la estructura y estilo del texto. Algunas implementaciones también han propuesto esquemas de pesado inspirados en tf-idf poniendo énfasis en el estilo y contenido de los textos.

%En el caso de los modelos basados en redes neuronales, la extracción de características es implícita en el modelo.

Con respecto a los \textbf{algoritmos de clasificación} existe una gran cantidad de enfoques, siendo lo más común el algoritmo de máquinas de soporte vectorial (SVM). Un punto importante a destacar, es que a partir del año 2018 se han presentado algunas propuestas utilizando aprendizaje profundo. Sin embargo, hasta la fecha no han podido superar a los algoritmos tradicionales. 
La primera vez que un enfoque de aprendizaje profundo, concretamente una arquitectura CNN, aparece entre los primeros lugares (lugar once) fue en la conferencia PAN 2019 \citep{Rangel2019}. Es importante notar, que ha diferencia del enfoque tradicional, en el caso de los modelos basados en redes neuronales, la extracción de características está implícita en su arquitectura.

\subsection{Detección de trastornos mentales en redes sociales}
%¿El lenguaje contiene información acerca del estado mental de una persona? Es una pregunta que es de interés por muchos investigadores  por mas de 70 años.
%Con respecto al perfilado de condiciones médicas podemos considerar a las competencias del eRISK. Este tipo de competencias están orientadas a la detección temprana de enfermedades mentales \citep{}.
Uno de los primeros estudios en la detección de depresión mediante medios automáticos \citep{rude2004language} encontró que los participantes deprimidos utilizan más palabras negativas y el uso del pronombre ``yo" ( \textit{``I"} en inglés) más que los participantes no deprimidos. 

%Derivado de la investigación psicológica, \citep{wang2013depression} utilizo caracteristicas tales como el uso de pronombres en primera persona, la interacción social de los usuarios y comportamientos en micro-blogs.  
Moviéndose a la investigación en redes sociales \citep{de2013predicting} empleo \textit{crowdsourcing} (una forma de colaboración empleando a múltiples personas a través de internet) para obtener un conjunto de usuarios de Twitter quienes reportaron ser diagnosticados con depresión clínica, en este trabajo demostraron el uso potencial de Twitter como una fuente de información para medir signos de depresión mayor en individuos. Encontraron que los individuos con depresión muestran baja actividad social, emociones negativas, auto atención, expresión de preocupaciones médicas y relacionales, además de pensamientos religiosos. Estos atributos fueron considerados para construir un clasificador SVM alcanzando una precisión de 70\%. Estos resultados demostraron la factibilidad de detectar la depresión en redes sociales. 

Sin embargo, la creación de colecciones de documentos para abordar este tipo de problemas es costosa y difícil. Bajo estas condiciones y debido al particular interés en el perfilado de características de comportamiento \citep{kumar2018aggression} y trastornos mentales  \citep{de2013predicting}, se han desarrollado conferencias y competencias específicas.
Una de ellas es la conferencia eRISK \citep{Losada2018}. El principal objetivo de este foro es la detección temprana de un trastorno a través de los historiales de comunicación de un usuario en blogs. Independientemente del enfoque de detección temprana también es de interés tratar la detección
considerando todo el historial de un usuario como un solo documento. 

El eRISK 2018 presentó dos tareas: detección de depresión y detección de anorexia. En ambos casos se trata de un problema de clasificación no temática con datos desbalanceados.


\subsubsection{eRisk 2018}

En la edición eRisk 2017 \citep{losada2017erisk, Losada2018}, los organizadores construyeron un conjunto de datos con publicaciones de usuarios deprimidos y no de-primidos extraídos de la red social Reddit. En eRisk 2018 se complementó el conjunto original con más usuarios y se agregó la tarea de detección de anorexia. Para abordar estas tareas se evaluaron un total de 45 contribuciones de diferentes instituciones, algunas de las propuestas dieron un tratamiento estándar experimentado con diferentes características, LDA, n-gramas de palabras y diferentes esquemas de pesado \citep{cacheda2018analysis, almeida2017detecting, ortega2018peimex}. Mientras que muy pocos utilizaron enfoques de aprendizaje profundo: \citep{trotzek2018word, wang2018neural, liu2018tua1}.

El equipo TUA1 \citep{liu2018tua1} además de presentar un modelo construido con una SVM Lineal, pesado \textit{tf-idf} y normalización $l2$, también construyeron un modelo basado en una arquitectura compuesta de una red CNN -que actúa como extractor de características- y una LSTM. En su configuración experimental utilizaron una longitud de entrada de 2000 tokens, 64 filtros para la red CNN de tamaño 5, \textit{MaxPooling} de tamaño 4, un factor de 0.25 para \textit{dropout} y \textit{Relu} como función de activación. Para la fase de entrenamiento eligieron entropía cruzada binaria para la función de pérdida y el optimizador \textit{Adam}. Mediante este modelo neuronal se obtuvieron 0.29 de \textit{F1}, en la tarea de depresión y 0.36 para la detección de anorexia. 

Los investigadores que conformaron el equipo TBS \citep{wang2018neural} abordaron las tareas como un problema de clasificación de oraciones y presentaron un modelo basado en CNN en combinación con un pesado \textit{tf-idf}; obteniendo 0.26 de \textit{F1} para la detección de depresión y 0.67 para anorexia.

El equipo ganador \citep{trotzek2018word} FHDO-BCSG presento 5 modelos de clasificación diferentes, para la detección de depresión. El mejor modelo fue un ensamble de bolsas de palabras BOW, con diferentes tipos de pesado y n-gramas, el algoritmo de clasificación utilizado fue regresión logística utilizando un peso modificado para cada clase para incrementar el costo de los falsos negativos; obteniendo 0.64 de \textit{F1} para depresión y 0.81 para la detección de anorexia. 
Además presentaron un modelo basado en una \textit{CNN} utilizando vectores \textit{FastText} de 300 dimensiones, entrenados con documentos extraídos de un corpus de 1.37 billones de comentarios en Reddit, una longitud de entrada de 100 tokens, una capa de convolución, 100 filtros con altura igual a 2 y con un ancho correspondiente al tamaño de los vectores de palabras, max pooling de tamaño 1 y CReLu como función de activación; resultando en un vector de 200 dimensiones por documento que es propagado a través de cuatro capas totalmente conectadas. El entrenamiento fue realizado utilizando el optimizador \textit{Adam} para minimizar la entropía cruzada, mediante un tamaño de batch de 10,000 documentos de 100 palabras  y una taza de aprendizaje de $1e-4$ durante 30 épocas. Este modelo logró obtener una puntuación \textit{F1} de 0.54 para la detección de depresión y 0.81 para anorexia. Agregando características extraídas manualmente lograron mejorar la puntuación \textit{F1} a 0.85 para la detección de anorexia.

En general la tarea de detección de depresión fue la más difícil y de un total de 45 modelos evaluados la puntuación \textit{F1} promedio fue de 0.42 mientras que para anorexia 0.56 indicando que aún falta mucho por mejorar en estas tareas. 



\section{Aumento de datos}

El aprendizaje profundo típicamente requiere grandes cantidades de datos etiquetados para tener éxito. El aumento de datos promete resolver el problema de la necesidad de más datos etiquetados, básicamente consiste en aplicar una serie de transformaciones a un ejemplo original para obtener un nuevo dato a partir de éste.

El término \textbf{aumento de datos} se refiere a métodos para construir una optimización iterativa o algoritmos de muestreo mediante la introducción de datos no observados o variables latentes \citep{van2001art}. La idea del aumento de datos nació en problemas de \textbf{datos incompletos}, como una forma de completar las celdas faltantes en una tabla de contingencia balanceada \citep{dempster1977maximum}. El aumento de datos automático es mayoritariamente utilizado en tareas relacionadas a visión computacional y ayudan a realizar un entrenamiento más robusto particularmente cuando el tamaño de los datos es pequeño. 


Derivado del estudio del estado del arte en aumento de datos, las técnicas de aumento de datos se pueden clasificar en dos categorías (no restringidas a un solo dominio): aquellas que se basan en aprendizaje supervisado y las que utilizan un enfoque semi-supervisado. Los basados en un enfoque supervisado crean muestras nuevas a partir de datos previamente etiquetados y las que utilizan un enfoque semi-supervisado obtienen ejemplos totalmente nuevos en base un modelo supervisado, supervisado débil o heurísticas conociendo la naturaleza de los datos.


\subsection{Aumento de datos supervisado} El objetivo es crear un nuevo y realista conjunto de entrenamiento aplicando una transformación a la entrada de un ejemplo. Conservando la etiqueta original del ejemplo. Formalmente, sea $q(\hat{x}|x)$ la transformación de aumento de la cual podemos extraer  ejemplos aumentados $\hat{x}$ basado en un ejemplo original $x$. Para que una transformación de aumento sea válida es requerido que cualquier ejemplo $\hat{x} \sim q(\hat{x}|x)$ extraído de la distribución comparta la misma etiqueta de verdad que $x$, es decir $y(\hat{x})=y(x)$. Dada una transformación de aumento válida, simplemente se pude minimizar la probabilidad negativa de los ejemplos aumentados \citep{xie2019unsupervised}. 

El aumento de datos supervisado puede ser equivalentemente visto como construir un conjunto aumentado etiquetado del conjunto original y entrenar el modelo en el conjunto aumentado. El punto crítico es como diseñar esa transformación, en la literatura podemos encontrar dos grupos de algoritmos para \textbf{crear} ejemplos de entrenamiento adicionales: los que operan \textbf{a nivel estructural}, los cuales crean transformaciones en un ejemplo (imagen, cadena de caracteres, texto, etc.) \citep{zhong2017random}, y \textbf{sobre muestreo sintético} creando ejemplos adicionales a nivel características es decir en un espacio vectorial\citep{chawla2002smote}. 

\subsection{Aumento de datos semi-supervisado}

Estos métodos tienen como característica general el aprender un modelo inicial para posteriormente etiquetar datos nuevos obtenidos de algún dominio similar y re-entrenar el modelo con estos datos nuevos. Tomando la definición de \citep{xie2019unsupervised}, la forma general de estos trabajos puede ser resumida como sigue:

\begin{itemize}
    \item Dada una entrada $x$, se calcula la distribución $p_\theta (y|x)$ dado $x$ y una versión con ruido $p_\theta (y|x, \epsilon)$ mediante la introducción de un pequeño ruido $\epsilon$. El ruido puede ser aplicado a $x$ o estados ocultos.
    \item Minimizar una métrica de divergencia entre las dos distribuciones $D (p_\theta (y|x) || p_\theta (y|x, \epsilon))$.
\end{itemize}

Este procedimiento forza el modelo a ser insensible al ruido $\epsilon$ y suave con respecto a los cambios en el espacio de entrada. Desde otra perspectiva, minimizando la pérdida de consistencia gradualmente se propaga la información de la etiqueta de ejemplos etiquetados a ejemplos no etiquetados \citep{Miyato2019}.


\subsection{Aumento de datos en clasificación de textos}

El aumento de datos ha sido ampliamente utilizado en tareas de visión computacional \citep{cubuk2019autoaugment}, pero menos en tareas de procesamiento de lenguaje natural. En años recientes ha crecido el interés por proponer diversas técnicas para el aumento de datos en la clasificación de textos, a continuación se mencionan algunos de los métodos más relevantes para este trabajo.

\subsubsection{Basados en métodos semi-supervisados}

Datos con ruido: \citep{hedderich2018training} propusieron una capa de ruido que es agregada a una arquitectura de red neuronal. Lo que permite modelar el ruido y entrenar una combinación de datos limpios y con ruido, para simular escenarios de pocos recursos el entrenamiento fue ralizado con diferentes tamaños de datos limpios, variando desde un 1\% del conjunto original hasta un 10\% (equivalentes de 407 ejemplos y 20,362 respectivamente). Comprobando que en un contexto de bajos recursos reduciendo el conjunto original hasta un 1\%, en la tarea de reconocimiento de entidades nombradas (NER), la clasificación puede mejorar en términos de \textit{F1} en promedio hasta 10 puntos en F1 mediante el uso adicional de datos con ruido y manejando el ruido. Variando el tamaño del conjunto original a un 10\% la ganancia obtenida no se observa por lo que se llega a la conclusión de que un 10\% de datos limpios puede ser suficiente para entrenar el modelo y el ruido adicional puede perjudicar al modelo.

Reinforced Co-Training: \citep{wu2018reinforced}, este método utiliza el algoritmo Q-learning para aprender una política de selección de datos y entonces explotar esta política para co-entrenar clasificadores automáticamente. Realizaron experimentos en la detección de \textit{Clickbait}; este término se refiere a aquellos encabezados con el objetivo de atraer la atención del lector, pero los documentos usualmente tienen menos relevancia con los encabezados correspondientes. El etiquetado de este tipo de datos consume mucho tiempo y labor. En esta tarea lograron mejorar 3 puntos en términos de la métrica \textit{F1} en comparación con el modelo base entrenado en forma supervisada.

\citep{han2019neural} propusieron una técnica de aumento de datos la cual consiste en incorporar ejemplos nuevos al conjunto de entrenamiento mediante un etiquetado basado en la búsqueda de similitudes relacionales en millones de tweets no etiquetados. Realizaron experimentos para la detección de rumores en redes sociales, logrando incrementar en promedio la métrica \textit{F1} entre 9 y 12 puntos en comparación a no realizar aumento de datos.

UDA \citep{xie2019unsupervised}: Es una propuesta híbrida la cual consiste en utilizar métodos existentes de aumento de datos, reemplazo de sinónimos y traducción inversa, para aumentar datos etiquetados y no etiquetados. Mediante el entrenamiento fino del modelo no supervisado BERT lograron aproximar el error de clasificación en 4 conjuntos de datos para la clasificación de opiniones con un margen de un punto porcentual, en comparación con el modelo entrenado en el conjunto completo de datos etiquetados. Con esto se logró comprobar que aún existe una brecha por rebasar cuando se comparan los métodos supervisados con los semi-supervisados.

Por lo general los esquemas para realizar aumento de datos de forma semi-supervisa han requerido de modelos complejos para poder implementarse. Si bien los resultados son prometedores y comparables al estado del arte, no han logrado superar el estado del arte basado en modelos supervisados.

\subsubsection{Basados en aprendizaje supervisado}

\citep{zhang2015character} presentaron una exploración empírica de redes convolucionales a nivel carácter. Construyeron conjuntos de datos aumentados para la clasificación de opiniones, mediante el reemplazo de palabras por sus sinónimos utilizando un tesauros. Llegando a reducir el error de clasificación en 1\% menos en comparación con el estado del arte, agregando aumento de datos en cuatro de ocho conjuntos de datos.


Aumento de datos contextual \citep{kobayashi2018contextual}: Asumen que el sentido de las oraciones no cambia incluso si las palabras en las oraciones son reemplazadas por otras palabras con relaciones paradigmáticas. Este método, estocásticamente reemplaza palabras con otras palabras que son predichas por un modelo de lenguaje bi-direccional. Además proponen un modelo de lenguaje condicionado a la etiqueta que permite al modelo aumentar oraciones considerando la información de la etiqueta. Mediante experimentos en 6 conjuntos de datos de clasificación de textos logran mejorar la exactitud en 1\% en comparación a no realizar aumento de datos y menor a 1\% en comparación con el remplazo de sinónimos.

EDA \citep{wei2019eda}: se presenta como una alternativa simple y escalable en comparación con métodos de aumentos de datos basados en redes neuronales, EDA consiste de una combinación de cuatro operaciones a nivel palabra: reemplazo de sinónimos, inserción aleatoria, intercambio aleatorio y eliminación aleatoria. En cinco tareas de clasificación, muestran que es posible mejorar  el rendimiento en redes convolucionales y recurrentes, alcanzado entre un 1 y 2\% en comparación de modelos sin aumento de datos.

Paráfrasis neuronal \citep{kumar2019submodular}: Este trabajo propone un método para obtener paráfrasis neuronales mediante el modelo seq2seq, a diferencia de otros modelos para generar paráfrasis este método busca un balance entre la diversidad y la fidelidad de las oraciones generadas; para esto proponen optimizar un función que combine estos dos factores. Los autores evaluaron su propuesta para la clasificación de intención utilizando una red LSTM y regresión logística, obteniendo una mejora de 3\% en exactitud sobre el método base que es no realizar aumento de datos y 2\% sobre reemplazo de sinónimos.

Traducción de temas \citep{zhang2019integrating}: Este método traduce todas las palabras reemplazables de una oración a otras clases objetivo. Esta búsqueda de relaciones de similitud se realiza utilizando aritmética de vectores. Realizaron diversos experimentos para la clasificación de documentos mediante \textit{zero-shoot text clasification}, esta técnica de clasificación consiste en ser capaz de predecir categorías no vistas en la fase de entrenamiento. Mediante un esquema controlado de poco recursos logran obtener ganancias de 1 a 8\% en términos de exactitud, en comparación a no realizar aumento de datos.


\subsection{Discusión del trabajo relacionado}

Al revisar la literatura de los métodos de aumento de datos basados en un enfoque supervisado, podemos observar que son un tanto complejos y en muchos casos, bajo un esquema de experimentación controlando la cantidad de datos etiquetados disponibles, no logran superar a los modelos supervisados.

Todos los trabajos hasta ahora encontrados en la literatura de aumento de datos mediante un enfoque supervisado, están enfocados a la clasificación de textos cortos o clasificación temática,  pero ni una enfocada a tareas de perfilado de autor o demostrado ser efectivos en conjuntos desbalanceados. En algunos casos como EDA el reemplazo es totalmente aleatorio o la estructura del documento se corrompe al incorporar operaciones de eliminación sobre las palabras, en otros como el reemplazo de sinónimos no siempre se asegura que la palabra a reemplazar pertenezca a la misma categoría que la palabra original. Los trabajos que respetan la estructura y diversidad de la oración original están basados en modelos de redes neuronales pero es difícil hacerlos escalables. En la tabla \ref{table:sup_meth} se presentan las principales características de los diferentes enfoques supervisados relevantes para este trabajo en comparación con el método propuesto. La propuesta de \cite{kumar2019submodular} puede considerarse que se respeta el estilo  contenido de los textos mediante la realización de una paráfrasis neuronal, pero este enfoque tiene la desventaja de utilizar un conjunto de datos externos para aprender a realizar la paráfrasis y considerando que se cuente con este recurso, el tiempo tomado para realizar un paráfrasis neuronal o predecir palabras mediante un modelo de lenguaje hace que el método no sea escalable.

En el caso de perfilado de autor, es necesario que los nuevos ejemplos aumentados respeten tanto el estilo (i.e., la estructura original) como el contenido del texto, por lo que en este trabajo se proponen métodos de aumento de datos que consideren el estilo y contenido del documento original; considerando estilo como la forma o modo de expresar el contenido, siendo el contenido el tema o mensaje a transmitir.

Los resultados hasta ahora alcanzados muestran un beneficio del uso del aumento de datos, no obstante, estos beneficios son aún modestos. Por otro lado, las técnicas simples de aumento de datos a nivel palabra han demostrado ser efectivas y escalables, y obtienen resultados comparables a técnicas complejas como  la paráfrasis neuronal o modelos de lenguaje.

\begin{table}[!ht]
\caption{Comparación del método propuesto con el estado del arte en aumento de datos para clasificación de textos.} \label{table:sup_meth}

\resizebox{\columnwidth}{!}{
\begin{tabular}{llllllll}
\hline
\rowcolor[HTML]{EFEFEF} 
Metodo             & Clasificación        & \begin{tabular}[c]{@{}l@{}}Datos \\ desbalanceados\end{tabular} & \begin{tabular}[c]{@{}l@{}}Recurso \\ externo\end{tabular} & \begin{tabular}[c]{@{}l@{}}Parafrasis neuronal \\ o \\ modelo de lenguaje\end{tabular} & Estilo      & Contenido \\ \hline
Zhan2015           & Minería de opiniones              & No                                                              & Tesauro                                                    & No                                                                    & No          & Si                                                                                 \\ \hline
Kobayashi2018      & Temática             & No                                                              & Glove                                                      & Si                                                                    &      No & Si                                                                                  \\ \hline
EDA                & Minería de opiniones              & No                                                              & Tesauro                                                    & No                                                                    & No          & Si                                                                                  \\ \hline
Kumar2019          & Dialogo            & No                                                              & \begin{tabular}[c]{@{}l@{}}Datos \\ alineados\end{tabular} & Si                                                                    & Si          & Si                                                                                  \\ \hline
Zhang2019          & Tematica             & No                                                              & Glove                                                      & No                                                                    & Si          & No                                                                                  \\ \hline
\textbf{Propuesta} & \textbf{Perfilado} & \textbf{Si}                                                     & Glove                                                      & \textbf{No}                                                                    & \textbf{Si} & \textbf{Si}                                                                         \\ \hline
\end{tabular}

}
\end{table}











