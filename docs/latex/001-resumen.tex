\chapter{Resumen}
%El aumento de datos consiste en crear un nuevo conjunto realista mediante una serie de transformaciones a un ejemplo. En este trabajo se proponen diferentes estrategias para el aumento de datos en perfilado de situaciones médicas. 

%El aumento de datos es una técnica mayormente empleada en la clasificación de imágenes y recientemente a crecido un interés para aplicarlo en la clasificación textos. 
%Se han propuesto diversas estrategias para el aumento de datos en tareas de clasificación de textos pero ni una específicamente para tareas relacionadas al perfilado de autor.

%Esta tesis propone un esquema general para el aumento de datos con diferentes estrategias de selección y reemplazo de palabras, principalmente enfocándose a las relaciones de similitud de las palabras. 

%Gracias a los experimentos realizados fue posible concluir que el aumento de datos propuesto puede mejorar la predicción en tareas relacionadas al perfilado de autor, en comparación con no realizar aumento de datos y algoritmos existen para el aumento de datos en la clasificación textos.

%Por ultimo, este trabajo es el primero en demostrar el efecto del aumento de datos en tareas relacionadas al perfilado de autor.


Para resolver las tareas de perfilado de autor, la mayoría de los trabajos existentes se han enfocado en utilizar algoritmos de aprendizaje computacional en combinación con diferentes técnicas para extraer características. La obtención de dichas características requiere un análisis riguroso y en muchos casos es necesaria la intervención de expertos en el tema. Sin embargo existen técnicas de aprendizaje computacional más complejas como las redes neuronales en donde la extracción de características se realiza de forma automática mediante una serie de abstracciones.

La principal motivación para el uso de redes neuronales en perfilado de autor, es debido al increíble éxito del aprendizaje profundo en tareas complejas del procesamiento del  lenguaje natural. De acuerdo al estado del arte en la última conferencia del PAN@CLEF los equipos con mejores resultados utilizaron técnicas tradicionales de aprendizaje. Así también en las tareas del ERISK el mejor sistema se construyó extrayendo características en combinación con un ensamble de bolsas de palabras y diferentes clasificadores. Lo que se ha podido observar en los diferentes reportes de estas conferencias es, que los modelos de aprendizaje basados en redes neuronales no han tenido el éxito esperado. 

Uno de los principales problemas dentro es la cantidad de datos etiquetados con que se cuenta y se hace más notable cuando se utilizan modelos de aprendizaje profundo. En el caso de perfilado de autor la obtención de estos datos etiquetados manualmente consumen mucho tiempo, son costosos, además se podría comprometer a problemas legales debido al uso de datos personales. 

Dada esta problemática este trabajo presenta un estudio sobre el efecto de agregar documentos nuevos, generados artificialmente, mediante aumento de datos a nivel estructural, al  conjunto de entrenamiento original y el efecto que tiene en los algoritmos de redes neuronales aplicados en tareas relacionadas al perfilado de autor. Para ello, en esta tesis se propone un esquema general para el aumento de datos con diferentes estrategias de selección y reemplazo de palabras, principalmente enfocándose a las relaciones de similitud de las palabras. 

Gracias a los experimentos realizados fue posible concluir que el aumento de datos propuesto puede mejorar la predicción en tareas relacionadas al perfilado de autor, en comparación con no realizar aumento de datos y algoritmos existen para el aumento de datos en la clasificación textos.

\chapter{Abstract}






