\section{Resultados}

En la tabla \ref{table:resultados} se presentan los resultados de los experimentos mediante el promedio  de la métrica $F1$ calculada en base a la clase de interés (la clase positiva).  En esta tabla se comparan los métodos propuestos; reemplazo mediante relaciones quivalentes, relaciones contrarias y restricción mediante la selección de características $\chi^2$, contra la línea base (sin aumento de datos) y los métodos de referencia: (i) sobre muestreo, (ii) utilizando un tesauro y (iii)  selección sin restricción.

En primer lugar se compara la línea base (no realizar aumento de datos) contra los diferentes métodos de aumento de datos, en donde se puede observar que la mayoría de los métodos superan esta línea base, a excepción del modelo basado en SVM-C para el conjunto de depresión. Comparando los métodos de referencia y  propuestos los métodos que sobre salen son:  el basado en \textit{restricción $\chi^2$} que en promedio obtiene 0.53 en \textit{F1} para el conjunto de depresión y 0.79 para el conjunto de anorexia, además del método basado en \textit{relaciones contrarias} obteniendo el mejor promedio F1 para el conjunto de anorexia con 0.80 y una puntuación muy similar que los métodos de referencia para el conjunto de depresión. 

Con respecto a los algoritmos el aumento de datos obtiene mejores resultados en la red recurrente \textit{Bi-LSTM} en comparación con la red convolucional. Por otra parte en los algoritmos lineales se observa un gran incremento en el modelo SVM obteniendo mejores resultados en comparación con el algoritmo SVM-C que considera el desbalance de las clases. 

En resumen el mejor valor obtenido para el conjunto de depresión es mediante el método propuesto \textit{Restricción Chi2} con una puntuación F1 de 0.56 y para el conjunto de anorexia mediante el método propuesto \textit{Relación contraria} alcanzando un F1 de 0.83.


\begin{table}[ht]
\caption{Resultados en términos de la métrica F1, la variable $n$ indica la magnitud del aumento en el conjunto original.} \label{table:resultados}

\begin{center}

\resizebox{\columnwidth}{!}{%

\begin{tabular}{lllrlrrrrr}
\hline
\textbf{Conjunto de datos} &                                         & \textbf{Bi-LSTM}                       & \multicolumn{1}{l}{}                          & \textbf{CNN}                           & \multicolumn{1}{l}{}                          & \multicolumn{1}{l}{\textbf{SVM}}             & \multicolumn{1}{l}{}                          & \multicolumn{1}{l}{\textbf{SVM-C}}           & \multicolumn{1}{l}{}                          \\ \hline
                           & \textit{Método}                         & \textit{F1}                            & \multicolumn{1}{l}{\textit{n}}                & \textit{F1}                            & \multicolumn{1}{l}{\textit{n}}                & \multicolumn{1}{l}{F1}              & \multicolumn{1}{l}{n}                         & \multicolumn{1}{l}{F1}              & \multicolumn{1}{l}{n}                         \\ \hline
Depresión 2018             & \cellcolor[HTML]{C0C0C0}Sin aumento     & \cellcolor[HTML]{C0C0C0}47$\pm$3          & \multicolumn{1}{l}{\cellcolor[HTML]{C0C0C0}-} & \cellcolor[HTML]{C0C0C0}45$\pm$5          & \multicolumn{1}{l}{\cellcolor[HTML]{C0C0C0}-} & \cellcolor[HTML]{C0C0C0}16          & \multicolumn{1}{l}{\cellcolor[HTML]{C0C0C0}-} & \cellcolor[HTML]{C0C0C0}50          & \multicolumn{1}{l}{\cellcolor[HTML]{C0C0C0}-} \\ \hline
                           & \cellcolor[HTML]{EFEFEF}Over            & \cellcolor[HTML]{EFEFEF}48$\pm$3          & \cellcolor[HTML]{EFEFEF}2                     & \cellcolor[HTML]{EFEFEF}51$\pm$2          & \cellcolor[HTML]{EFEFEF}2                     & \cellcolor[HTML]{EFEFEF}51          & \cellcolor[HTML]{EFEFEF}4                     & \cellcolor[HTML]{EFEFEF}51          & \cellcolor[HTML]{EFEFEF}9                     \\ \hline
                           & \cellcolor[HTML]{EFEFEF}Tesauro         & \cellcolor[HTML]{EFEFEF}45$\pm$3          & \cellcolor[HTML]{EFEFEF}4                     & \cellcolor[HTML]{EFEFEF}52$\pm$2          & \cellcolor[HTML]{EFEFEF}2                     & \cellcolor[HTML]{EFEFEF}50          & \cellcolor[HTML]{EFEFEF}10                    & \cellcolor[HTML]{EFEFEF}48          & \cellcolor[HTML]{EFEFEF}1                     \\ \hline
                           & \cellcolor[HTML]{EFEFEF}Sin restricción & \cellcolor[HTML]{EFEFEF}47$\pm$4          & \cellcolor[HTML]{EFEFEF}1                     & \cellcolor[HTML]{EFEFEF}52$\pm$1 & \cellcolor[HTML]{EFEFEF}2                     & \cellcolor[HTML]{EFEFEF}53 & \cellcolor[HTML]{EFEFEF}10                    & \cellcolor[HTML]{EFEFEF}50          & \cellcolor[HTML]{EFEFEF}3                     \\ \hline
                           & Restricción $\chi^2$                         & \textbf{49$\pm$3}                         & 9                                             & \textbf{53$\pm$1}                                  & 1                                             & \textbf{53}                         & 10                                            & 50                                  & 1                                             \\ \hline
                           & Relación Positiva                       & 45$\pm$5                                  & 4                                             & 48$\pm$4                                  & 1                                             & 41                                  & 8                                             & 49                                  & 1                                             \\ \hline
                           & Relación Contraria                      & 35$\pm$6                                  & 1                                             & 49$\pm$1                                  & 10                                            & 52                                  & 8                                             & \textbf{51}                         & 1                                             \\ \hline
\textit{Depresión 2019}    & \cellcolor[HTML]{C0C0C0}Sin aumento     & \cellcolor[HTML]{C0C0C0}74$\pm$8          & \multicolumn{1}{l}{\cellcolor[HTML]{C0C0C0}-} & \cellcolor[HTML]{C0C0C0}81$\pm$3          & \multicolumn{1}{l}{\cellcolor[HTML]{C0C0C0}-} & \cellcolor[HTML]{C0C0C0}0           & \multicolumn{1}{l}{\cellcolor[HTML]{C0C0C0}-} & \cellcolor[HTML]{C0C0C0}50          & \multicolumn{1}{l}{\cellcolor[HTML]{C0C0C0}-} \\ \hline
                           & \cellcolor[HTML]{EFEFEF}Over            & \cellcolor[HTML]{EFEFEF}81$\pm$13         & \cellcolor[HTML]{EFEFEF}1                     & \cellcolor[HTML]{EFEFEF}85$\pm$3          & \cellcolor[HTML]{EFEFEF}2                     & \cellcolor[HTML]{EFEFEF}13          & \cellcolor[HTML]{EFEFEF}9            & \cellcolor[HTML]{EFEFEF}\textbf{50} & \cellcolor[HTML]{EFEFEF}1                     \\ \hline
                           & \cellcolor[HTML]{EFEFEF}Tesauro         & \cellcolor[HTML]{EFEFEF}80$\pm$5          & \cellcolor[HTML]{EFEFEF}7                     & \cellcolor[HTML]{EFEFEF}\textbf{88$\pm$2} & \cellcolor[HTML]{EFEFEF}7                     & \cellcolor[HTML]{EFEFEF}10          & \cellcolor[HTML]{EFEFEF}10                    & \cellcolor[HTML]{EFEFEF}49          & \cellcolor[HTML]{EFEFEF}1                     \\ \hline
                           & \cellcolor[HTML]{EFEFEF}Sin restricción & \cellcolor[HTML]{EFEFEF}74$\pm$10         & \cellcolor[HTML]{EFEFEF}1                     & \cellcolor[HTML]{EFEFEF}85$\pm$2          & \cellcolor[HTML]{EFEFEF}2                     & \cellcolor[HTML]{EFEFEF}10          & \cellcolor[HTML]{EFEFEF}7                     & \cellcolor[HTML]{EFEFEF}47          & \cellcolor[HTML]{EFEFEF}1                     \\ \hline
                           & Restricción $\chi^2$                         & \textbf{88$\pm$4}                         & 1                                             & 87$\pm$3                                  & 1                                             & 10                                  & 8                                             & 47                                  & 1                                             \\ \hline
                           & Relación Positiva                       & 78$\pm$15                                 & 1                                             & \textbf{88$\pm$2}                         & 6                                             & 6                                   & 8                                             & 46                                  & 1                                             \\ \hline
                           & Relación Contraria                      & 65$\pm$7                                  & 9                                             & 82$\pm$3                                  & 9                                             & \textbf{31}                         & 4                                             & 15                                  & 5                                             \\ \hline
\textit{Anorexia}          & \cellcolor[HTML]{C0C0C0}Sin aumento     & \cellcolor[HTML]{C0C0C0}75$\pm$3          & \multicolumn{1}{l}{\cellcolor[HTML]{C0C0C0}-} & \cellcolor[HTML]{C0C0C0}80$\pm$0          & \multicolumn{1}{l}{\cellcolor[HTML]{C0C0C0}-} & \cellcolor[HTML]{C0C0C0}67          & \multicolumn{1}{l}{\cellcolor[HTML]{C0C0C0}-} & \cellcolor[HTML]{C0C0C0}72          & \multicolumn{1}{l}{\cellcolor[HTML]{C0C0C0}-} \\ \hline
                           & \cellcolor[HTML]{EFEFEF}Over            & \cellcolor[HTML]{EFEFEF}76$\pm$4 & \cellcolor[HTML]{EFEFEF}1                     & \cellcolor[HTML]{EFEFEF}80$\pm$1          & \cellcolor[HTML]{EFEFEF}1                     & \cellcolor[HTML]{EFEFEF}77          & \cellcolor[HTML]{EFEFEF}7                     & \cellcolor[HTML]{EFEFEF}75          & \cellcolor[HTML]{EFEFEF}6                     \\ \hline
                           & \cellcolor[HTML]{EFEFEF}Tesauro         & \cellcolor[HTML]{EFEFEF}74$\pm$5 & \cellcolor[HTML]{EFEFEF}3                     & \cellcolor[HTML]{EFEFEF}80$\pm$1          & \cellcolor[HTML]{EFEFEF}4                     & \cellcolor[HTML]{EFEFEF}76          & \cellcolor[HTML]{EFEFEF}2                     & \cellcolor[HTML]{EFEFEF}75          & \cellcolor[HTML]{EFEFEF}3                     \\ \hline
                           & \cellcolor[HTML]{EFEFEF}Sin restricción & \cellcolor[HTML]{EFEFEF}76$\pm$4          & \cellcolor[HTML]{EFEFEF}4                     & \cellcolor[HTML]{EFEFEF}81$\pm$1          & \cellcolor[HTML]{EFEFEF}1                     & \cellcolor[HTML]{EFEFEF}78          & \cellcolor[HTML]{EFEFEF}6                     & \cellcolor[HTML]{EFEFEF}\textbf{78 }& \cellcolor[HTML]{EFEFEF}6                     \\ \hline
                           & Restricción $\chi^2$                         & \textbf{76$\pm$3}                                  & 1                                             & \textbf{81$\pm$1}                         & 1                                             & 78                                  & 5                                             & 77                                  & 1                                             \\ \hline
                           & Relación Positiva                       & 76$\pm$3                                  & 1                                             & 80$\pm$1                         & 1                                             & 78                                  & 6                                             & 76                                  & 1                                             \\ \hline
                           & Relación Contraria                      & 74$\pm$1                                  & 1                                             & 80$\pm$1                                  & 1                                             & \textbf{81}                         & 8                                             & 75                                  & 5                                             \\ \hline
\end{tabular}%

}
\end{center}
\end{table}




